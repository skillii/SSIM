% **************************************************************************************************
% ** SPSC Report and Thesis Template
% **************************************************************************************************
%
% ***** Authors *****
% Daniel Arnitz, Paul Meissner, Stefan Petrik
% Signal Processing and Speech Communication Laboratory (SPSC)
% Graz University of Technology (TU Graz), Austria
%
% ***** Changelog *****
%
% ***** Todo *****
%
% **************************************************************************************************



\documentclass[%
a4paper,% !!! ATTENTION: geometry package below !!!
\Twosided,% !!! ATTENTION: geometry package below !!!
openany,% begin chapters with new right page (openright) or don't care (openany)
11pt,%
fleqn,% equations not centered, but on the left side
tablecaptionbelow,% captions below tables
% titlepage,% use title
pointlessnumbers,% do not generate point at the end of section numbers (e.g. 1.4.5 instead of 1.4.5.)
final,%
]{scrreprt}% (KOMA)

\usepackage[paper=a4paper,\Twosided,%
textheight=246mm,%
textwidth=160mm,%
heightrounded=true,% round textheight to multiple of lines (avoids overfull vboxes)
ignoreall=true,% do not include header, footer, and margins in calculations
marginparsep=5pt,% marginpar only used for signs (centered), thus only small sep. needed
marginparwidth=10mm,% prevent margin notes to be out of page
hmarginratio=2:1,% set margin ration (inner:outer for twoside) - (2:3 is default)
]{geometry}%


% master
\usepackage{ifthen}% for optional parts
\usepackage[latin1]{inputenc}% German special characters
\ifthenelse{\equal{\DocumentLanguage}{en}}{\usepackage[USenglish]{babel}}{}%
\ifthenelse{\equal{\DocumentLanguage}{de}}{\usepackage[ngerman]{babel}}{}%
\usepackage[%
headtopline,plainheadtopline,% activate all lines (header and footer)
headsepline,plainheadsepline,%
footsepline,plainfootsepline,%
footbotline,plainfootbotline,%
automark% auto update \..mark
]{scrpage2}% (KOMA)
\usepackage{makeidx}% used to make an index directory
\usepackage[]{caption}% customize captions
\usepackage{multicol}%
\usepackage[stable,bottom,hang,splitrule,multiple,symbol*]{footmisc}% customize footnotes


% text
\usepackage{varioref}% improved references
\usepackage{color}% e.g., for color boxes
\usepackage{rotating}% to rotate objects
\usepackage{gensymb}% symbols (perthousand, Celsius, ...)
\usepackage[right]{eurosym}% euro symbol on the right side (51 EUR)
\usepackage[normalem]{ulem}% cross-out, strike-out, underlines (normalem: keep \emph italic)
%\usepackage[safe]{textcomp}% loading in safe mode to avoid problems (see LaTeX companion)
%\usepackage[geometry,misc]{ifsym}% technical symbols
\usepackage{remreset}%\@removefromreset commands (e.g., for continuous footnote numbering)
\usepackage[%
breaklinks=true,% allow line break in links
colorlinks=true,% if false: framed link
linkcolor=black,anchorcolor=black,citecolor=black,filecolor=black,%
menucolor=black,urlcolor=black]{hyperref}% hyperlinks for references


% math
\usepackage{amsmath,amssymb,amstext,bm} % use math packages
\usepackage{mathcomp}% symbols (perthousand, ...) in math mode


% graphics
\usepackage{graphicx}% use simple graphics
\usepackage{subfigure}% subfigures (a),(b),(c)... within figures
\usepackage{flafter}% place floats always after reference
\usepackage{placeins}% preventing floats from crossing a barrier
\usepackage{float}% to place floats !HERE!
\usepackage{psfrag}% replace text in eps figures


% tables
\usepackage{hhline}% hline doesn't work with colored columns, so using hhline
\usepackage{longtable}% for tables longer than one page
\usepackage{dcolumn}% for number alignment in tables
\usepackage{colortbl}% color in tables


% listings
%\usepackage{alltt}% verbatim environment with commands available
\usepackage{listings}% program code listings


% other
%\usepackage{layout}% graphical page layout (spacings)
\usepackage{xspace}% add space after macros if not followed by punctuation character
\makeindex% used for index creation

